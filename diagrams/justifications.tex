\documentclass{article}
\begin{document}
\begin{itemize}
	\item Note:
		This is a consultatory database and the user is not intended to add items to the database.
		Thus, it will not include user-added teams.
	\item LearnsByBreeding:
		This is separate from the Learns table because learning by breeding is a special case
		that has special requirements.
		Ternary Relationship between Move, a Mother Pokemon, and a Father Pokemon.
		This is because for some moves, a Pokemon can learn it when the father,
		which is not necessarily the same species as the child, knows the move and breeds
		with a mother who is of the same species as the child.
		Not all Pokemon can learn moves this way, not all moves can be learned this way, 
		and not all Pokemon can pass on moves to their children.
		Furthermore, many Pokemon can pass on many moves to many different Pokemon.
	\item Learns:
		All Pokemon can learn at least one move and all moves can be learned.
		The same move can be learned in several different ways, so the learn method
		is included as a part of the primary key to distinguish more than one 
		instance of Learns between the same Pokemon and move.
	\item IsType: 
		All moves have a type, and all types are assigned to at least one move.
		Thus, there is mandatory participation from both sides.
		A move can only have one type, but one type can be assigned to many moves.
	\item Effectiveness:
		All moves have an effectiveness against all types.
		Since all types need to participate in this relation with every other type,
		it is a many-to-many relationship with total participation.
	\item HasTypes:
		A Pokemon always has one or two types. This means that all Pokemon need to participate
		in this relationship. The ??? type does not participate in this relationship
		because no Pokemon has that type. There can be multiple Pokemon
		with the same type and multiple types can assigned to one Pokemon.
	\item EvolvesFrom:
		Not all Pokemon evolve and not all Pokemon evolve from another Pokemon,
		hence, there is not total participation.
		A Pokemon can only evolve from one other Pokemon,
		but many Pokemon can evolve from the same Pokemon.
		For instance, Flareon and Jolteon both evolve from Eevee.
	\item FoundAt:
		Not all Pokemon can be found in the wild. Jirachi, for instance,
		is only available through special events and cannot be caught normally.
		Not all locations have Pokemon that can be found there either. 
		For instance, most towns have no wild Pokemon.
		The same Pokemon can be found in many different locations and
		most locations where Pokemon can be caught have multiple 
		different Pokemon that can be caught there.
	\item FoughtAt:
		We are only tracking data about Non-Player Character trainers.
		All trainers can be fought at at least one location.
		Not all locations have trainers. Some trainers, such as your
		rival in the game are fought in several different locations
		throughout the game.
		Most locations that have trainers have more than one trainer.
	\item uses:
		Team is a weak entity connected to a trainer through uses.
		All teams belong to exactly one trainer and one trainer,
		through rematches, may have more than one team.
		All NPC trainers in the game have a team,
		otherwise, they would not be a trainer.
	\item TeamMember:
		All Teams have at least one Pokemon and can have more.
		Not all Pokemon are in a trainer's team. Deoxys, for instance, is not
		in any trainer's team.
		One team can have multiple different Pokemon as well.
		Since, it is possible for a team to have more than one 
		of the same Pokemon, it is necessary to include the
		party index: MemberID as a way to distinguish 
		individual party members.
		It should be noted that we did not include 
		the moves that a team member knows because
		the data was not present in any open-source
		repository that we could find.
		We used Bulbapedia because it is under
		a creative commons license and it lists
		moves for team members of ``boss'' trainers,
		but not for most. For consistency's sake, we decided to include no
		team member's moves.
\end{itemize}
\end{document}