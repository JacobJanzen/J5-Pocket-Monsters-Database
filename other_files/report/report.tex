\documentclass{article}
\usepackage{microtype}
\usepackage[binary-units]{siunitx}
\usepackage{todonotes}
\usepackage{booktabs}
\usepackage{listings}
\usepackage{xcolor}

\title{Pok\'emon Emerald Database}
\author{
    Jacob Janzen\\
    \texttt{janzenj2@myumanitoba.ca} \and 
    Daniel La Rocque\\
    \texttt{larocq17@myumanitoba.ca} \and 
    Jared Webber\\
    \texttt{webberj1@myumanitoba.ca} \and 
    Colin Johnson\\
    \texttt{johns233@myumanitoba.ca}
}

\begin{document}
\maketitle
\section{Data Summary}
Pok\'emon was fresh on our minds when we began the project because our instructor, Adam Pazdor,
had brought up his interest in the game several times during class. We very quickly realized that
a database containing information about Pok\'emon would be a great way to have lots of data with
very interesting relationships.

We ultimately decided to constrain our idea to one specific game in the franchise while building
our data model because we wanted to track data about non-player characters and locations in the
games which would quickly get very complicated when discussing multiple games. It would introduce
even more complications when considering the fact that many relationships between entities would
change depending on the specific game. To solve this, we chose one specific game rather 
arbitrarily: Pok\'emon Emerald.

The Data consists of the set of Pok\'emon that are present in Pok\'emon Emerald, the locations, 
moves, and non-player \emph{trainers} in the game, as well as the \emph{types} that Pok\'emon 
and moves can have. We tracked the methods in which each Pok\'emon can learn each move as well as 
the location of all non-player characters and the locations where each Pok\'emon can be found.

In the end, we had a \SI{1.4}{\mega\byte} database. Our largest table, \verb+Learns+, had 15280
records while our smallest table, \verb+HM+, had only 8. In total, there were 24547 records in
the database.

\section{Data Model}
We gathered data based on the data model we created, which was created from brainstorming every
type of data we could gather about the game that we deemed interesting and then linked together
in every way that we could think of.

Most of the challenges in creating the data model were in fine tuning the details after working
out exactly which data we would use. For instance, originally the \verb+Learns+ table contained
every move that every Pok\'emon could learn without any distinction of method. Later, we decided
to add the method and split it into one relationship for each method in which a Pok\'emon could
learn a move. This added a lot of unnecessary complications to the data set and made queries
far more difficult than they needed to be. Thus, we combined them back into a single relationship 
with a method attribute being a primary key on the relationship. Unfortunately, learning a move by
breeding is a little more complicated than other methods of learning a move. It is dependent not
only upon the Pok\'emon learning the move and the move itself, but also the father of the 
Pok\'emon learning the move. Therefore, we separated that ternary relationship between two
Pok\'emon and move from the rest of the learn methods.

Another challenge that came up in the project was the relationship between a trainer and the 
Pok\'emon that they used in their team. Originally, we had \verb+Team+ as a weak entity that 
depended on \verb+Trainer+ and \verb+TeamMember+ as a weak entity that depended on \verb+Team+.
This two-layer weak entity added a lot of unnecessary complication, so we changed 
\verb+TeamMember+ to be a relationship instead which has an ID as a part of its primary key.

A tricky participation ratio was surrounding the \verb+HasTypes+ relationship. The ??? type
which existed in Pok\'emon Emerald was a type only used for the move \emph{Curse}.
Thus, we had only partial participation from \verb+Type+ to \verb+Pokemon+ and total participation
from the other side because every Pok\'emon has at least one type and can have two.

A tricky cardinality ratio is through the \verb+EvolvesFrom+ relationship. While generally, when
a Pok\'emon evolves, it can only evolve into one Pok\'emon, there are rare situations where
one Pok\'emon can evolve into multiple different ones. For instance, \emph{Eevee} can evolve into 
\emph{Vaporeon}, \emph{Jolteon}, \emph{Flareon}, \emph{Espeon}, or \emph{Umbreon} depending on the
specific circumstances that it evolves under. This relationship does not work in the opposite 
direction. There are no cases where one Pok\'emon evolves from two different Pok\'emon in 
Pok\'emon Emerald.

\section{Database Summary}
\begin{center}
    \begin{tabular}{ccc}
        \toprule
        Table & Cardinality & Arity\\
        \midrule
        Pokemon & 386 & 17\\
        Abilities & 610 & 2\\
        EggGroups & 504 & 2\\
        Location & 106 & 1\\
        Move & 354 & 6\\
        Trainer & 493 & 3\\
        Type & 18 & 2\\
        FoughtAt & 520 & 2\\
        FoundAt & 647 & 6\\
        HasTypes & 557 & 2\\
        HM & 8 & 2\\
        Learns & 15280 & 3\\
        LearnsByBreeding & 2007 & 3\\
        Team & 887 & 3\\
        TeamMember & 1795 & 6\\
        TM & 50 & 2\\
        Effectiveness & 324 & 3\\
        \bottomrule
    \end{tabular}
\end{center}

\section{Queries}
\begin{itemize}
    \item The Pok\'emon in a given egg group.
    \item The Pok\'emon a given Pok\'emon can breed with.
    \item The locations a Pok\'emon may be found at.
    \item The locations a Pok\'emon may be found at and the method which they can be found with.
    \item The locations a Pok\'emon of a given type can be found.
    \item The locations that a Pok\'emon with two given types can be found.
    \item The locations with a given trainer.
    \item The locations where a certain trainer class can be fought.
    \item The locations where a certain Pok\'emon of at least a certain level can be found.
    \item List the moves that are learnt by all Pok\'emon.
    \item Moves that are learnt by all Pok\'emon and the method by which they are learnt.
    \item Moves that are learnt by a given Pok\'emon.
    \item Moves that are learnt by a given Pok\'emon and the method by which they are learnt.
    \item Moves that have a given effectiveness against a given type.
    \item Moves that are super-effective against a given Pok\'emon.
    \item Moves that are neutral against a given Pok\'emon.
    \item Moves that are not-very-effective against a given Pok\'emon.
    \item Moves that a given Pok\'emon is immune to.
    \item All moves' effectiveness against a given Pok\'emon.
    \item List all status moves.
    \item Ways a given Pok\'emon can learn a given move.
    \item  Moves of a given type that a given Pok\'emon can learn.
    \item Moves of a given type that a given Pok\'emon can learn and the method by which they are
        learnt.
    \item Moves a Pok\'emon learns through breeding.
    \item Moves a Pok\'emon can learn through breeding with a given father.
    \item All Pok\'emon names.
    \item All Pok\'emon hatch times.
    \item All move names.
    \item Pok\'emon and their Pok\'edex position.
    \item All trainer data.
    \item Stats of a given Pok\'emon.
    \item Evolutions of a given Pok\'emon.
    \item Pok\'emon with a given move.
    \item Pok\'emon from each type with the highest value in a given stat.
    \item Pok\'emon from each type with the lowest value in a given stat.
    \item Pok\'emon having a minimum value of a given stat.
    \item Pok\'emon having a maximum value of a given stat.
    \item Pok\'emon which can be caught at a given location.
    \item Pok\'emon which can be caught at a given location from a given encounter.
    \item Pok\'emon which can learn a move that is super-effective on a given Pok\'emon.
    \item Pok\'emon that can be caught at a given location that can learn a move that is 
        super-effective on a given Pok\'emon.
    \item Pok\'emon that a given move is super-effective against.
    \item Pok\'emon that a given move is neutral against.
    \item Pok\'emon that a given move is non-very-effective against.
    \item Pok\'emon that are immune to a given move.
    \item Effects a given move has on a Pok\'emon.
    \item All abilities Pok\'emon have.
    \item Pok\'emon with a given ability.
    \item Pok\'emon of a given type that can learn moves of a given type.
    \item Pok\'emon of a given type that can learn moves of a given type and the methods by which
        they are learnt.
    \item All teams of a given trainer.
    \item All teams with a given Pok\'emon.
    \item All team with a minimum level.
    \item All teams with a maximum level.
    \item Number of Pok\'emon per type.
    \item Types with physical damage.
    \item Types with special damage.
\end{itemize}

\section{Interface}
The project was created with an Electron front-end written in JavaScript with Vue.js. The back-end
is a Flask application written in Python. The back-end runs the queries which are requested 
through API calls from the front-end using Axios as an HTTP client. 
The application can be started with the following on Linux/macOS:
\begin{verbatim}
    .../3380_Project $ ./run-linux.sh
\end{verbatim}
and on Windows:
\begin{verbatim}
    ...\3380_Project>./run-windows.bat
\end{verbatim}
On both, you can then visit \verb+http://localhost:8080/+ to see the application.
To close the application, press CTRL+C to close the front-end and then again to close the 
back-end.

The interface is fairly simple. There are four tabs at the top of the screen. Starting from left
to right, they are: a homepage containing information about us, a link to our repository, and
details about how to navigate through the application; a copy of this report; our 
Enhanced Entity Relation Diagram; and finally, the database interaction page.

Interacting with the database is quite simple. You can start by clicking the drop-down menu saying
``What do you want to know?'' Afterwards, you can start typing to search for a specific query or 
to filter down the list, or just scroll down the list of queries until you find your desired 
query. After selecting a query, it may open up new drop-down menus to select specific parameters
of the query. When all the fields have been selected, you can click ``QUERY!'' to execute the 
query and the results will appear on the page. 
If you want to save a CSV file containing the results of the query, you can click download. To 
execute another query, click ``NEW QUERY.''

\end{document}
