\documentclass{article}
\usepackage{microtype}
\usepackage[binary-units]{siunitx}
\usepackage{todonotes}
\usepackage{booktabs}

\title{Pok\'emon Emerald Database}
\author{
    Jacob Janzen\\
    \texttt{janzenj2@myumanitoba.ca} \and 
    Daniel La Rocque\\
    \texttt{larocq17@myumanitoba.ca} \and 
    Jared Webber\\
    \texttt{webberj1@myumanitoba.ca} \and 
    Colin Johnson\\
    \texttt{johns233@myumanitoba.ca}
}

\begin{document}
\maketitle
\section{Data Summary}
Pok\'emon was fresh on our minds when we began the project because our instructor, Adam Pazdor,
had brought up his interest in the game several times during class. We very quickly realized that
a database containing information about Pok\'emon would be a great way to have lots of data with
very interesting relationships.

We ultimately decided to constrain our idea to one specific game in the franchise while building
our data model because we wanted to track data about non-player characters and locations in the
games which would quickly get very complicated when discussing multiple games. It would introduce
even more complications when considering the fact that many relationships between entities would
change depending on the specific game. To solve this, we chose one specific game rather 
arbitrarily: Pok\'emon Emerald.

The Data consists of the set of Pok\'emon that are present in Pok\'emon Emerald, the locations, 
moves, and non-player \emph{trainers} in the game, as well as the \emph{types} that Pok\'emon 
and moves can have. We tracked the methods in which each Pok\'emon can learn each move as well as 
the location of all non-player characters and the locations where each Pok\'emon can be found.

In the end, we had a \SI{1.4}{\mega\byte} database. Our largest table, \verb+Learns+, had 15280
records while our smallest table, \verb+HM+, had only 8. In total, there were 24547 records in
the database.

\section{Data Model}
We gathered data based on the data model we created, which was created from brainstorming every
type of data we could gather about the game that we deemed interesting and then linked together
in every way that we could think of.

Most of the challenges in creating the data model were in fine tuning the details after working
out exactly which data we would use. For instance, originally the \verb+Learns+ table contained
every move that every Pok\'emon could learn without any distinction of method. Later, we decided
to add the method and split it into one relationship for each method in which a Pok\'emon could
learn a move. This added a lot of unnecessary complications to the data set and made queries
far more difficult than they needed to be. Thus, we combined them back into a single relationship 
with a method attribute being a primary key on the relationship. Unfortunately, learning a move by
breeding is a little more complicated than other methods of learning a move. It is dependent not
only upon the Pok\'emon learning the move and the move itself, but also the father of the 
Pok\'emon learning the move. Therefore, we separated that ternary relationship between two
Pok\'emon and move from the rest of the learn methods.

Another challenge that came up in the project was the relationship between a trainer and the 
Pok\'emon that they used in their team. Originally, we had \verb+Team+ as a weak entity that 
depended on \verb+Trainer+ and \verb+TeamMember+ as a weak entity that depended on \verb+Team+.
This two-layer weak entity added a lot of unnecessary complication, so we changed 
\verb+TeamMember+ to be a relationship instead which has an ID as a part of its primary key.

A tricky participation ratio was surrounding the \verb+HasTypes+ relationship. The ??? type
which existed in Pok\'emon Emerald was a type only used for the move \emph{Curse}.
Thus, we had only partial participation from \verb+Type+ to \verb+Pokemon+ and total participation
from the other side because every Pok\'emon has at least one type and can have two.

A tricky cardinality ratio is through the \verb+EvolvesFrom+ relationship. While generally, when
a Pok\'emon evolves, it can only evolve into one Pok\'emon, there are rare situations where
one Pok\'emon can evolve into multiple different ones. For instance, \emph{Eevee} can evolve into 
\emph{Vaporeon}, \emph{Jolteon}, \emph{Flareon}, \emph{Espeon}, or \emph{Umbreon} depending on the
specific circumstances that it evolves under. This relationship does not work in the opposite 
direction. There are no cases where one Pok\'emon evolves from two different Pok\'emon in 
Pok\'emon Emerald.

\section{Database Summary}
\begin{center}
    \begin{tabular}{ccc}
        \toprule
        Table & Cardinality & Arity\\
        \midrule
        Pokemon & 386 & 17\\
        Abilities & 610 & 2\\
        EggGroups & 504 & 2\\
        Location & 106 & 1\\
        Move & 354 & 6\\
        Trainer & 493 & 3\\
        Type & 18 & 2\\
        FoughtAt & 520 & 2\\
        FoundAt & 647 & 6\\
        HasTypes & 557 & 2\\
        HM & 8 & 2\\
        Learns & 15280 & 3\\
        LearnsByBreeding & 2007 & 3\\
        Team & 887 & 3\\
        TeamMember & 1795 & 6\\
        TM & 50 & 2\\
        Effectiveness & 324 & 3\\
        \bottomrule
    \end{tabular}
\end{center}

\section{Queries}

\section{Interface}
\todo{add details about how to set up}

\end{document}
